\documentclass[11pt]{article}

    \usepackage[breakable]{tcolorbox}
    \usepackage{parskip} % Stop auto-indenting (to mimic markdown behaviour)
    

    % Basic figure setup, for now with no caption control since it's done
    % automatically by Pandoc (which extracts ![](path) syntax from Markdown).
    \usepackage{graphicx}
    % Keep aspect ratio if custom image width or height is specified
    \setkeys{Gin}{keepaspectratio}
    % Maintain compatibility with old templates. Remove in nbconvert 6.0
    \let\Oldincludegraphics\includegraphics
    % Ensure that by default, figures have no caption (until we provide a
    % proper Figure object with a Caption API and a way to capture that
    % in the conversion process - todo).
    \usepackage{caption}
    \DeclareCaptionFormat{nocaption}{}
    \captionsetup{format=nocaption,aboveskip=0pt,belowskip=0pt}

    \usepackage{float}
    \floatplacement{figure}{H} % forces figures to be placed at the correct location
    \usepackage{xcolor} % Allow colors to be defined
    \usepackage{enumerate} % Needed for markdown enumerations to work
    \usepackage{geometry} % Used to adjust the document margins
    \usepackage{amsmath} % Equations
    \usepackage{amssymb} % Equations
    \usepackage{textcomp} % defines textquotesingle
    % Hack from http://tex.stackexchange.com/a/47451/13684:
    \AtBeginDocument{%
        \def\PYZsq{\textquotesingle}% Upright quotes in Pygmentized code
    }
    \usepackage{upquote} % Upright quotes for verbatim code
    \usepackage{eurosym} % defines \euro

    \usepackage{iftex}
    \ifPDFTeX
        \usepackage[T1]{fontenc}
        \IfFileExists{alphabeta.sty}{
              \usepackage{alphabeta}
          }{
              \usepackage[mathletters]{ucs}
              \usepackage[utf8x]{inputenc}
          }
    \else
        \usepackage{fontspec}
        \usepackage{unicode-math}
    \fi

    \usepackage{fancyvrb} % verbatim replacement that allows latex
    \usepackage{grffile} % extends the file name processing of package graphics
                         % to support a larger range
    \makeatletter % fix for old versions of grffile with XeLaTeX
    \@ifpackagelater{grffile}{2019/11/01}
    {
      % Do nothing on new versions
    }
    {
      \def\Gread@@xetex#1{%
        \IfFileExists{"\Gin@base".bb}%
        {\Gread@eps{\Gin@base.bb}}%
        {\Gread@@xetex@aux#1}%
      }
    }
    \makeatother
    \usepackage[Export]{adjustbox} % Used to constrain images to a maximum size
    \adjustboxset{max size={0.9\linewidth}{0.9\paperheight}}

    % The hyperref package gives us a pdf with properly built
    % internal navigation ('pdf bookmarks' for the table of contents,
    % internal cross-reference links, web links for URLs, etc.)
    \usepackage{hyperref}
    % The default LaTeX title has an obnoxious amount of whitespace. By default,
    % titling removes some of it. It also provides customization options.
    \usepackage{titling}
    \usepackage{longtable} % longtable support required by pandoc >1.10
    \usepackage{booktabs}  % table support for pandoc > 1.12.2
    \usepackage{array}     % table support for pandoc >= 2.11.3
    \usepackage{calc}      % table minipage width calculation for pandoc >= 2.11.1
    \usepackage[inline]{enumitem} % IRkernel/repr support (it uses the enumerate* environment)
    \usepackage[normalem]{ulem} % ulem is needed to support strikethroughs (\sout)
                                % normalem makes italics be italics, not underlines
    \usepackage{soul}      % strikethrough (\st) support for pandoc >= 3.0.0
    \usepackage{mathrsfs}
    

    
    % Colors for the hyperref package
    \definecolor{urlcolor}{rgb}{0,.145,.698}
    \definecolor{linkcolor}{rgb}{.71,0.21,0.01}
    \definecolor{citecolor}{rgb}{.12,.54,.11}

    % ANSI colors
    \definecolor{ansi-black}{HTML}{3E424D}
    \definecolor{ansi-black-intense}{HTML}{282C36}
    \definecolor{ansi-red}{HTML}{E75C58}
    \definecolor{ansi-red-intense}{HTML}{B22B31}
    \definecolor{ansi-green}{HTML}{00A250}
    \definecolor{ansi-green-intense}{HTML}{007427}
    \definecolor{ansi-yellow}{HTML}{DDB62B}
    \definecolor{ansi-yellow-intense}{HTML}{B27D12}
    \definecolor{ansi-blue}{HTML}{208FFB}
    \definecolor{ansi-blue-intense}{HTML}{0065CA}
    \definecolor{ansi-magenta}{HTML}{D160C4}
    \definecolor{ansi-magenta-intense}{HTML}{A03196}
    \definecolor{ansi-cyan}{HTML}{60C6C8}
    \definecolor{ansi-cyan-intense}{HTML}{258F8F}
    \definecolor{ansi-white}{HTML}{C5C1B4}
    \definecolor{ansi-white-intense}{HTML}{A1A6B2}
    \definecolor{ansi-default-inverse-fg}{HTML}{FFFFFF}
    \definecolor{ansi-default-inverse-bg}{HTML}{000000}

    % common color for the border for error outputs.
    \definecolor{outerrorbackground}{HTML}{FFDFDF}

    % commands and environments needed by pandoc snippets
    % extracted from the output of `pandoc -s`
    \providecommand{\tightlist}{%
      \setlength{\itemsep}{0pt}\setlength{\parskip}{0pt}}
    \DefineVerbatimEnvironment{Highlighting}{Verbatim}{commandchars=\\\{\}}
    % Add ',fontsize=\small' for more characters per line
    \newenvironment{Shaded}{}{}
    \newcommand{\KeywordTok}[1]{\textcolor[rgb]{0.00,0.44,0.13}{\textbf{{#1}}}}
    \newcommand{\DataTypeTok}[1]{\textcolor[rgb]{0.56,0.13,0.00}{{#1}}}
    \newcommand{\DecValTok}[1]{\textcolor[rgb]{0.25,0.63,0.44}{{#1}}}
    \newcommand{\BaseNTok}[1]{\textcolor[rgb]{0.25,0.63,0.44}{{#1}}}
    \newcommand{\FloatTok}[1]{\textcolor[rgb]{0.25,0.63,0.44}{{#1}}}
    \newcommand{\CharTok}[1]{\textcolor[rgb]{0.25,0.44,0.63}{{#1}}}
    \newcommand{\StringTok}[1]{\textcolor[rgb]{0.25,0.44,0.63}{{#1}}}
    \newcommand{\CommentTok}[1]{\textcolor[rgb]{0.38,0.63,0.69}{\textit{{#1}}}}
    \newcommand{\OtherTok}[1]{\textcolor[rgb]{0.00,0.44,0.13}{{#1}}}
    \newcommand{\AlertTok}[1]{\textcolor[rgb]{1.00,0.00,0.00}{\textbf{{#1}}}}
    \newcommand{\FunctionTok}[1]{\textcolor[rgb]{0.02,0.16,0.49}{{#1}}}
    \newcommand{\RegionMarkerTok}[1]{{#1}}
    \newcommand{\ErrorTok}[1]{\textcolor[rgb]{1.00,0.00,0.00}{\textbf{{#1}}}}
    \newcommand{\NormalTok}[1]{{#1}}

    % Additional commands for more recent versions of Pandoc
    \newcommand{\ConstantTok}[1]{\textcolor[rgb]{0.53,0.00,0.00}{{#1}}}
    \newcommand{\SpecialCharTok}[1]{\textcolor[rgb]{0.25,0.44,0.63}{{#1}}}
    \newcommand{\VerbatimStringTok}[1]{\textcolor[rgb]{0.25,0.44,0.63}{{#1}}}
    \newcommand{\SpecialStringTok}[1]{\textcolor[rgb]{0.73,0.40,0.53}{{#1}}}
    \newcommand{\ImportTok}[1]{{#1}}
    \newcommand{\DocumentationTok}[1]{\textcolor[rgb]{0.73,0.13,0.13}{\textit{{#1}}}}
    \newcommand{\AnnotationTok}[1]{\textcolor[rgb]{0.38,0.63,0.69}{\textbf{\textit{{#1}}}}}
    \newcommand{\CommentVarTok}[1]{\textcolor[rgb]{0.38,0.63,0.69}{\textbf{\textit{{#1}}}}}
    \newcommand{\VariableTok}[1]{\textcolor[rgb]{0.10,0.09,0.49}{{#1}}}
    \newcommand{\ControlFlowTok}[1]{\textcolor[rgb]{0.00,0.44,0.13}{\textbf{{#1}}}}
    \newcommand{\OperatorTok}[1]{\textcolor[rgb]{0.40,0.40,0.40}{{#1}}}
    \newcommand{\BuiltInTok}[1]{{#1}}
    \newcommand{\ExtensionTok}[1]{{#1}}
    \newcommand{\PreprocessorTok}[1]{\textcolor[rgb]{0.74,0.48,0.00}{{#1}}}
    \newcommand{\AttributeTok}[1]{\textcolor[rgb]{0.49,0.56,0.16}{{#1}}}
    \newcommand{\InformationTok}[1]{\textcolor[rgb]{0.38,0.63,0.69}{\textbf{\textit{{#1}}}}}
    \newcommand{\WarningTok}[1]{\textcolor[rgb]{0.38,0.63,0.69}{\textbf{\textit{{#1}}}}}
    \makeatletter
    \newsavebox\pandoc@box
    \newcommand*\pandocbounded[1]{%
      \sbox\pandoc@box{#1}%
      % scaling factors for width and height
      \Gscale@div\@tempa\textheight{\dimexpr\ht\pandoc@box+\dp\pandoc@box\relax}%
      \Gscale@div\@tempb\linewidth{\wd\pandoc@box}%
      % select the smaller of both
      \ifdim\@tempb\p@<\@tempa\p@
        \let\@tempa\@tempb
      \fi
      % scaling accordingly (\@tempa < 1)
      \ifdim\@tempa\p@<\p@
        \scalebox{\@tempa}{\usebox\pandoc@box}%
      % scaling not needed, use as it is
      \else
        \usebox{\pandoc@box}%
      \fi
    }
    \makeatother

    % Define a nice break command that doesn't care if a line doesn't already
    % exist.
    \def\br{\hspace*{\fill} \\* }
    % Math Jax compatibility definitions
    \def\gt{>}
    \def\lt{<}
    \let\Oldtex\TeX
    \let\Oldlatex\LaTeX
    \renewcommand{\TeX}{\textrm{\Oldtex}}
    \renewcommand{\LaTeX}{\textrm{\Oldlatex}}
    % Document parameters
    % Document title
    \title{ThemePark\_PWA\_Workbook}
    
    
    
    
    
    
    
% Pygments definitions
\makeatletter
\def\PY@reset{\let\PY@it=\relax \let\PY@bf=\relax%
    \let\PY@ul=\relax \let\PY@tc=\relax%
    \let\PY@bc=\relax \let\PY@ff=\relax}
\def\PY@tok#1{\csname PY@tok@#1\endcsname}
\def\PY@toks#1+{\ifx\relax#1\empty\else%
    \PY@tok{#1}\expandafter\PY@toks\fi}
\def\PY@do#1{\PY@bc{\PY@tc{\PY@ul{%
    \PY@it{\PY@bf{\PY@ff{#1}}}}}}}
\def\PY#1#2{\PY@reset\PY@toks#1+\relax+\PY@do{#2}}

\@namedef{PY@tok@w}{\def\PY@tc##1{\textcolor[rgb]{0.73,0.73,0.73}{##1}}}
\@namedef{PY@tok@c}{\let\PY@it=\textit\def\PY@tc##1{\textcolor[rgb]{0.24,0.48,0.48}{##1}}}
\@namedef{PY@tok@cp}{\def\PY@tc##1{\textcolor[rgb]{0.61,0.40,0.00}{##1}}}
\@namedef{PY@tok@k}{\let\PY@bf=\textbf\def\PY@tc##1{\textcolor[rgb]{0.00,0.50,0.00}{##1}}}
\@namedef{PY@tok@kp}{\def\PY@tc##1{\textcolor[rgb]{0.00,0.50,0.00}{##1}}}
\@namedef{PY@tok@kt}{\def\PY@tc##1{\textcolor[rgb]{0.69,0.00,0.25}{##1}}}
\@namedef{PY@tok@o}{\def\PY@tc##1{\textcolor[rgb]{0.40,0.40,0.40}{##1}}}
\@namedef{PY@tok@ow}{\let\PY@bf=\textbf\def\PY@tc##1{\textcolor[rgb]{0.67,0.13,1.00}{##1}}}
\@namedef{PY@tok@nb}{\def\PY@tc##1{\textcolor[rgb]{0.00,0.50,0.00}{##1}}}
\@namedef{PY@tok@nf}{\def\PY@tc##1{\textcolor[rgb]{0.00,0.00,1.00}{##1}}}
\@namedef{PY@tok@nc}{\let\PY@bf=\textbf\def\PY@tc##1{\textcolor[rgb]{0.00,0.00,1.00}{##1}}}
\@namedef{PY@tok@nn}{\let\PY@bf=\textbf\def\PY@tc##1{\textcolor[rgb]{0.00,0.00,1.00}{##1}}}
\@namedef{PY@tok@ne}{\let\PY@bf=\textbf\def\PY@tc##1{\textcolor[rgb]{0.80,0.25,0.22}{##1}}}
\@namedef{PY@tok@nv}{\def\PY@tc##1{\textcolor[rgb]{0.10,0.09,0.49}{##1}}}
\@namedef{PY@tok@no}{\def\PY@tc##1{\textcolor[rgb]{0.53,0.00,0.00}{##1}}}
\@namedef{PY@tok@nl}{\def\PY@tc##1{\textcolor[rgb]{0.46,0.46,0.00}{##1}}}
\@namedef{PY@tok@ni}{\let\PY@bf=\textbf\def\PY@tc##1{\textcolor[rgb]{0.44,0.44,0.44}{##1}}}
\@namedef{PY@tok@na}{\def\PY@tc##1{\textcolor[rgb]{0.41,0.47,0.13}{##1}}}
\@namedef{PY@tok@nt}{\let\PY@bf=\textbf\def\PY@tc##1{\textcolor[rgb]{0.00,0.50,0.00}{##1}}}
\@namedef{PY@tok@nd}{\def\PY@tc##1{\textcolor[rgb]{0.67,0.13,1.00}{##1}}}
\@namedef{PY@tok@s}{\def\PY@tc##1{\textcolor[rgb]{0.73,0.13,0.13}{##1}}}
\@namedef{PY@tok@sd}{\let\PY@it=\textit\def\PY@tc##1{\textcolor[rgb]{0.73,0.13,0.13}{##1}}}
\@namedef{PY@tok@si}{\let\PY@bf=\textbf\def\PY@tc##1{\textcolor[rgb]{0.64,0.35,0.47}{##1}}}
\@namedef{PY@tok@se}{\let\PY@bf=\textbf\def\PY@tc##1{\textcolor[rgb]{0.67,0.36,0.12}{##1}}}
\@namedef{PY@tok@sr}{\def\PY@tc##1{\textcolor[rgb]{0.64,0.35,0.47}{##1}}}
\@namedef{PY@tok@ss}{\def\PY@tc##1{\textcolor[rgb]{0.10,0.09,0.49}{##1}}}
\@namedef{PY@tok@sx}{\def\PY@tc##1{\textcolor[rgb]{0.00,0.50,0.00}{##1}}}
\@namedef{PY@tok@m}{\def\PY@tc##1{\textcolor[rgb]{0.40,0.40,0.40}{##1}}}
\@namedef{PY@tok@gh}{\let\PY@bf=\textbf\def\PY@tc##1{\textcolor[rgb]{0.00,0.00,0.50}{##1}}}
\@namedef{PY@tok@gu}{\let\PY@bf=\textbf\def\PY@tc##1{\textcolor[rgb]{0.50,0.00,0.50}{##1}}}
\@namedef{PY@tok@gd}{\def\PY@tc##1{\textcolor[rgb]{0.63,0.00,0.00}{##1}}}
\@namedef{PY@tok@gi}{\def\PY@tc##1{\textcolor[rgb]{0.00,0.52,0.00}{##1}}}
\@namedef{PY@tok@gr}{\def\PY@tc##1{\textcolor[rgb]{0.89,0.00,0.00}{##1}}}
\@namedef{PY@tok@ge}{\let\PY@it=\textit}
\@namedef{PY@tok@gs}{\let\PY@bf=\textbf}
\@namedef{PY@tok@ges}{\let\PY@bf=\textbf\let\PY@it=\textit}
\@namedef{PY@tok@gp}{\let\PY@bf=\textbf\def\PY@tc##1{\textcolor[rgb]{0.00,0.00,0.50}{##1}}}
\@namedef{PY@tok@go}{\def\PY@tc##1{\textcolor[rgb]{0.44,0.44,0.44}{##1}}}
\@namedef{PY@tok@gt}{\def\PY@tc##1{\textcolor[rgb]{0.00,0.27,0.87}{##1}}}
\@namedef{PY@tok@err}{\def\PY@bc##1{{\setlength{\fboxsep}{\string -\fboxrule}\fcolorbox[rgb]{1.00,0.00,0.00}{1,1,1}{\strut ##1}}}}
\@namedef{PY@tok@kc}{\let\PY@bf=\textbf\def\PY@tc##1{\textcolor[rgb]{0.00,0.50,0.00}{##1}}}
\@namedef{PY@tok@kd}{\let\PY@bf=\textbf\def\PY@tc##1{\textcolor[rgb]{0.00,0.50,0.00}{##1}}}
\@namedef{PY@tok@kn}{\let\PY@bf=\textbf\def\PY@tc##1{\textcolor[rgb]{0.00,0.50,0.00}{##1}}}
\@namedef{PY@tok@kr}{\let\PY@bf=\textbf\def\PY@tc##1{\textcolor[rgb]{0.00,0.50,0.00}{##1}}}
\@namedef{PY@tok@bp}{\def\PY@tc##1{\textcolor[rgb]{0.00,0.50,0.00}{##1}}}
\@namedef{PY@tok@fm}{\def\PY@tc##1{\textcolor[rgb]{0.00,0.00,1.00}{##1}}}
\@namedef{PY@tok@vc}{\def\PY@tc##1{\textcolor[rgb]{0.10,0.09,0.49}{##1}}}
\@namedef{PY@tok@vg}{\def\PY@tc##1{\textcolor[rgb]{0.10,0.09,0.49}{##1}}}
\@namedef{PY@tok@vi}{\def\PY@tc##1{\textcolor[rgb]{0.10,0.09,0.49}{##1}}}
\@namedef{PY@tok@vm}{\def\PY@tc##1{\textcolor[rgb]{0.10,0.09,0.49}{##1}}}
\@namedef{PY@tok@sa}{\def\PY@tc##1{\textcolor[rgb]{0.73,0.13,0.13}{##1}}}
\@namedef{PY@tok@sb}{\def\PY@tc##1{\textcolor[rgb]{0.73,0.13,0.13}{##1}}}
\@namedef{PY@tok@sc}{\def\PY@tc##1{\textcolor[rgb]{0.73,0.13,0.13}{##1}}}
\@namedef{PY@tok@dl}{\def\PY@tc##1{\textcolor[rgb]{0.73,0.13,0.13}{##1}}}
\@namedef{PY@tok@s2}{\def\PY@tc##1{\textcolor[rgb]{0.73,0.13,0.13}{##1}}}
\@namedef{PY@tok@sh}{\def\PY@tc##1{\textcolor[rgb]{0.73,0.13,0.13}{##1}}}
\@namedef{PY@tok@s1}{\def\PY@tc##1{\textcolor[rgb]{0.73,0.13,0.13}{##1}}}
\@namedef{PY@tok@mb}{\def\PY@tc##1{\textcolor[rgb]{0.40,0.40,0.40}{##1}}}
\@namedef{PY@tok@mf}{\def\PY@tc##1{\textcolor[rgb]{0.40,0.40,0.40}{##1}}}
\@namedef{PY@tok@mh}{\def\PY@tc##1{\textcolor[rgb]{0.40,0.40,0.40}{##1}}}
\@namedef{PY@tok@mi}{\def\PY@tc##1{\textcolor[rgb]{0.40,0.40,0.40}{##1}}}
\@namedef{PY@tok@il}{\def\PY@tc##1{\textcolor[rgb]{0.40,0.40,0.40}{##1}}}
\@namedef{PY@tok@mo}{\def\PY@tc##1{\textcolor[rgb]{0.40,0.40,0.40}{##1}}}
\@namedef{PY@tok@ch}{\let\PY@it=\textit\def\PY@tc##1{\textcolor[rgb]{0.24,0.48,0.48}{##1}}}
\@namedef{PY@tok@cm}{\let\PY@it=\textit\def\PY@tc##1{\textcolor[rgb]{0.24,0.48,0.48}{##1}}}
\@namedef{PY@tok@cpf}{\let\PY@it=\textit\def\PY@tc##1{\textcolor[rgb]{0.24,0.48,0.48}{##1}}}
\@namedef{PY@tok@c1}{\let\PY@it=\textit\def\PY@tc##1{\textcolor[rgb]{0.24,0.48,0.48}{##1}}}
\@namedef{PY@tok@cs}{\let\PY@it=\textit\def\PY@tc##1{\textcolor[rgb]{0.24,0.48,0.48}{##1}}}

\def\PYZbs{\char`\\}
\def\PYZus{\char`\_}
\def\PYZob{\char`\{}
\def\PYZcb{\char`\}}
\def\PYZca{\char`\^}
\def\PYZam{\char`\&}
\def\PYZlt{\char`\<}
\def\PYZgt{\char`\>}
\def\PYZsh{\char`\#}
\def\PYZpc{\char`\%}
\def\PYZdl{\char`\$}
\def\PYZhy{\char`\-}
\def\PYZsq{\char`\'}
\def\PYZdq{\char`\"}
\def\PYZti{\char`\~}
% for compatibility with earlier versions
\def\PYZat{@}
\def\PYZlb{[}
\def\PYZrb{]}
\makeatother


    % For linebreaks inside Verbatim environment from package fancyvrb.
    \makeatletter
        \newbox\Wrappedcontinuationbox
        \newbox\Wrappedvisiblespacebox
        \newcommand*\Wrappedvisiblespace {\textcolor{red}{\textvisiblespace}}
        \newcommand*\Wrappedcontinuationsymbol {\textcolor{red}{\llap{\tiny$\m@th\hookrightarrow$}}}
        \newcommand*\Wrappedcontinuationindent {3ex }
        \newcommand*\Wrappedafterbreak {\kern\Wrappedcontinuationindent\copy\Wrappedcontinuationbox}
        % Take advantage of the already applied Pygments mark-up to insert
        % potential linebreaks for TeX processing.
        %        {, <, #, %, $, ' and ": go to next line.
        %        _, }, ^, &, >, - and ~: stay at end of broken line.
        % Use of \textquotesingle for straight quote.
        \newcommand*\Wrappedbreaksatspecials {%
            \def\PYGZus{\discretionary{\char`\_}{\Wrappedafterbreak}{\char`\_}}%
            \def\PYGZob{\discretionary{}{\Wrappedafterbreak\char`\{}{\char`\{}}%
            \def\PYGZcb{\discretionary{\char`\}}{\Wrappedafterbreak}{\char`\}}}%
            \def\PYGZca{\discretionary{\char`\^}{\Wrappedafterbreak}{\char`\^}}%
            \def\PYGZam{\discretionary{\char`\&}{\Wrappedafterbreak}{\char`\&}}%
            \def\PYGZlt{\discretionary{}{\Wrappedafterbreak\char`\<}{\char`\<}}%
            \def\PYGZgt{\discretionary{\char`\>}{\Wrappedafterbreak}{\char`\>}}%
            \def\PYGZsh{\discretionary{}{\Wrappedafterbreak\char`\#}{\char`\#}}%
            \def\PYGZpc{\discretionary{}{\Wrappedafterbreak\char`\%}{\char`\%}}%
            \def\PYGZdl{\discretionary{}{\Wrappedafterbreak\char`\$}{\char`\$}}%
            \def\PYGZhy{\discretionary{\char`\-}{\Wrappedafterbreak}{\char`\-}}%
            \def\PYGZsq{\discretionary{}{\Wrappedafterbreak\textquotesingle}{\textquotesingle}}%
            \def\PYGZdq{\discretionary{}{\Wrappedafterbreak\char`\"}{\char`\"}}%
            \def\PYGZti{\discretionary{\char`\~}{\Wrappedafterbreak}{\char`\~}}%
        }
        % Some characters . , ; ? ! / are not pygmentized.
        % This macro makes them "active" and they will insert potential linebreaks
        \newcommand*\Wrappedbreaksatpunct {%
            \lccode`\~`\.\lowercase{\def~}{\discretionary{\hbox{\char`\.}}{\Wrappedafterbreak}{\hbox{\char`\.}}}%
            \lccode`\~`\,\lowercase{\def~}{\discretionary{\hbox{\char`\,}}{\Wrappedafterbreak}{\hbox{\char`\,}}}%
            \lccode`\~`\;\lowercase{\def~}{\discretionary{\hbox{\char`\;}}{\Wrappedafterbreak}{\hbox{\char`\;}}}%
            \lccode`\~`\:\lowercase{\def~}{\discretionary{\hbox{\char`\:}}{\Wrappedafterbreak}{\hbox{\char`\:}}}%
            \lccode`\~`\?\lowercase{\def~}{\discretionary{\hbox{\char`\?}}{\Wrappedafterbreak}{\hbox{\char`\?}}}%
            \lccode`\~`\!\lowercase{\def~}{\discretionary{\hbox{\char`\!}}{\Wrappedafterbreak}{\hbox{\char`\!}}}%
            \lccode`\~`\/\lowercase{\def~}{\discretionary{\hbox{\char`\/}}{\Wrappedafterbreak}{\hbox{\char`\/}}}%
            \catcode`\.\active
            \catcode`\,\active
            \catcode`\;\active
            \catcode`\:\active
            \catcode`\?\active
            \catcode`\!\active
            \catcode`\/\active
            \lccode`\~`\~
        }
    \makeatother

    \let\OriginalVerbatim=\Verbatim
    \makeatletter
    \renewcommand{\Verbatim}[1][1]{%
        %\parskip\z@skip
        \sbox\Wrappedcontinuationbox {\Wrappedcontinuationsymbol}%
        \sbox\Wrappedvisiblespacebox {\FV@SetupFont\Wrappedvisiblespace}%
        \def\FancyVerbFormatLine ##1{\hsize\linewidth
            \vtop{\raggedright\hyphenpenalty\z@\exhyphenpenalty\z@
                \doublehyphendemerits\z@\finalhyphendemerits\z@
                \strut ##1\strut}%
        }%
        % If the linebreak is at a space, the latter will be displayed as visible
        % space at end of first line, and a continuation symbol starts next line.
        % Stretch/shrink are however usually zero for typewriter font.
        \def\FV@Space {%
            \nobreak\hskip\z@ plus\fontdimen3\font minus\fontdimen4\font
            \discretionary{\copy\Wrappedvisiblespacebox}{\Wrappedafterbreak}
            {\kern\fontdimen2\font}%
        }%

        % Allow breaks at special characters using \PYG... macros.
        \Wrappedbreaksatspecials
        % Breaks at punctuation characters . , ; ? ! and / need catcode=\active
        \OriginalVerbatim[#1,codes*=\Wrappedbreaksatpunct]%
    }
    \makeatother

    % Exact colors from NB
    \definecolor{incolor}{HTML}{303F9F}
    \definecolor{outcolor}{HTML}{D84315}
    \definecolor{cellborder}{HTML}{CFCFCF}
    \definecolor{cellbackground}{HTML}{F7F7F7}

    % prompt
    \makeatletter
    \newcommand{\boxspacing}{\kern\kvtcb@left@rule\kern\kvtcb@boxsep}
    \makeatother
    \newcommand{\prompt}[4]{
        {\ttfamily\llap{{\color{#2}[#3]:\hspace{3pt}#4}}\vspace{-\baselineskip}}
    }
    

    
    % Prevent overflowing lines due to hard-to-break entities
    \sloppy
    % Setup hyperref package
    \hypersetup{
      breaklinks=true,  % so long urls are correctly broken across lines
      colorlinks=true,
      urlcolor=urlcolor,
      linkcolor=linkcolor,
      citecolor=citecolor,
      }
    % Slightly bigger margins than the latex defaults
    
    \geometry{verbose,tmargin=1in,bmargin=1in,lmargin=1in,rmargin=1in}
    
    

\begin{document}
    
    \maketitle
    
    

    
    \section{Theme Park Flask PWA
Workbook}\label{theme-park-flask-pwa-workbook}

This notebook teaches the design, structure, and deployment of the
\textbf{Theme Park Progressive Web App (PWA)} built with Flask.

We will cover: 1. Purpose and design logic 2. Directory tree and file
explanations 3. Steps to create the PWA 4. Going live: deployment
modifications and CI/CD

    \subsection{1. Introduction \& Purpose
(Markdown)}\label{introduction-purpose-markdown}

\paragraph{What is a progressive web application
(PWA)?}\label{what-is-a-progressive-web-application-pwa}

\paragraph{Why do we chose Flask for this
PWA}\label{why-do-we-chose-flask-for-this-pwa}

\paragraph{What is the Theme Park
Scenario?}\label{what-is-the-theme-park-scenario}

We are going to design and build a web based information system for a
Theme Park, which will be device UI responsive

    \section{Theme Park Flask PWA
Workbook}\label{theme-park-flask-pwa-workbook}

This notebook teaches the design, structure, and deployment of the
\textbf{Theme Park Progressive Web App (PWA)} built with Flask.

    \subsubsection{🎯 Activity 1:
Introduction}\label{activity-1-introduction}

\begin{itemize}
\tightlist
\item
  In your own words, explain what a PWA is.
\item
  Why are we using Flask for this project? ✅ Success criteria: mentions
  offline/installable features, lightweight framework, Theme Park
  scenario relevance.
\end{itemize}

    \subsection{1. Purpose and Design Logic}\label{purpose-and-design-logic}

The Theme Park PWA is designed to: - Provide a \textbf{secure,
classroom‑ready web app} using Flask. - Demonstrate \textbf{modern web
features}: offline support, service workers, manifest. - Teach
\textbf{database integration}: storing users, events, staff, customers,
bookings. - Showcase \textbf{authentication}: Google OAuth + JWT for API
access. - Model \textbf{DevOps practices}: migrations, testing, CI/CD,
containerization.

    \subsection{2. Project Structure (Directory Tree) and file
extensions}\label{project-structure-directory-tree-and-file-extensions}

    \begin{tcolorbox}[breakable, size=fbox, boxrule=1pt, pad at break*=1mm,colback=cellbackground, colframe=cellborder]
\prompt{In}{incolor}{ }{\boxspacing}
\begin{Verbatim}[commandchars=\\\{\}]
\PY{c+c1}{\PYZsh{}\PYZsh{}\PYZsh{} This will give you a mental map of the project before diving into the code :)}

\PY{n}{flask\PYZus{}pwa\PYZus{}app}\PY{o}{/}
\PY{err}{├}\PY{err}{─}\PY{err}{─} \PY{n}{app}\PY{o}{.}\PY{n}{py}                  \PY{c+c1}{\PYZsh{} Main Flask app: routes, auth, JWT, error handling}
\PY{err}{├}\PY{err}{─}\PY{err}{─} \PY{n}{models}\PY{o}{.}\PY{n}{py}               \PY{c+c1}{\PYZsh{} SQLAlchemy models (User, Event, Staff, Customer, Booking)}
\PY{err}{├}\PY{err}{─}\PY{err}{─} \PY{n}{import\PYZus{}csv}\PY{o}{.}\PY{n}{py}           \PY{c+c1}{\PYZsh{} Utility to bulk import sample CSV data}
\PY{err}{├}\PY{err}{─}\PY{err}{─} \PY{n}{themepark}\PY{o}{.}\PY{n}{db}            \PY{c+c1}{\PYZsh{} SQLite database (local dev)}
\PY{err}{├}\PY{err}{─}\PY{err}{─} \PY{n}{requirements}\PY{o}{.}\PY{n}{txt}        \PY{c+c1}{\PYZsh{} Production dependencies}
\PY{err}{├}\PY{err}{─}\PY{err}{─} \PY{n}{requirements}\PY{o}{\PYZhy{}}\PY{n}{dev}\PY{o}{.}\PY{n}{txt}    \PY{c+c1}{\PYZsh{} Dev/test dependencies}
\PY{err}{├}\PY{err}{─}\PY{err}{─} \PY{n}{static}\PY{o}{/}                 \PY{c+c1}{\PYZsh{} Static assets}
\PY{err}{│}   \PY{err}{├}\PY{err}{─}\PY{err}{─} \PY{n}{css}\PY{o}{/}\PY{n}{style}\PY{o}{.}\PY{n}{css}       \PY{c+c1}{\PYZsh{} Styling}
\PY{err}{│}   \PY{err}{├}\PY{err}{─}\PY{err}{─} \PY{n}{js}\PY{o}{/}\PY{n}{app}\PY{o}{.}\PY{n}{js}           \PY{c+c1}{\PYZsh{} Client\PYZhy{}side logic}
\PY{err}{│}   \PY{err}{├}\PY{err}{─}\PY{err}{─} \PY{n}{js}\PY{o}{/}\PY{n}{service}\PY{o}{\PYZhy{}}\PY{n}{worker}\PY{o}{.}\PY{n}{js} \PY{c+c1}{\PYZsh{} Offline caching}
\PY{err}{│}   \PY{err}{├}\PY{err}{─}\PY{err}{─} \PY{n}{icons}\PY{o}{/}              \PY{c+c1}{\PYZsh{} PWA icons}
\PY{err}{│}   \PY{err}{└}\PY{err}{─}\PY{err}{─} \PY{n}{manifest}\PY{o}{.}\PY{n}{json}       \PY{c+c1}{\PYZsh{} PWA metadata}
\PY{err}{├}\PY{err}{─}\PY{err}{─} \PY{n}{templates}\PY{o}{/}              \PY{c+c1}{\PYZsh{} Jinja2 templates}
\PY{err}{│}   \PY{err}{├}\PY{err}{─}\PY{err}{─} \PY{n}{base}\PY{o}{.}\PY{n}{html}           \PY{c+c1}{\PYZsh{} Layout with navbar/footer}
\PY{err}{│}   \PY{err}{├}\PY{err}{─}\PY{err}{─} \PY{n}{index}\PY{o}{.}\PY{n}{html}          \PY{c+c1}{\PYZsh{} Home page}
\PY{err}{│}   \PY{err}{├}\PY{err}{─}\PY{err}{─} \PY{n}{offline}\PY{o}{.}\PY{n}{html}        \PY{c+c1}{\PYZsh{} Offline fallback}
\PY{err}{│}   \PY{err}{├}\PY{err}{─}\PY{err}{─} \PY{n}{test\PYZus{}api}\PY{o}{.}\PY{n}{html}       \PY{c+c1}{\PYZsh{} API test page}
\PY{err}{│}   \PY{err}{├}\PY{err}{─}\PY{err}{─} \PY{n}{error}\PY{o}{.}\PY{n}{html}          \PY{c+c1}{\PYZsh{} Error handler}
\PY{err}{│}   \PY{err}{├}\PY{err}{─}\PY{err}{─} \PY{n}{login}\PY{o}{.}\PY{n}{html}          \PY{c+c1}{\PYZsh{} Google login page}
\PY{err}{│}   \PY{err}{├}\PY{err}{─}\PY{err}{─} \PY{n}{admin}\PY{o}{.}\PY{n}{html}          \PY{c+c1}{\PYZsh{} Admin dashboard}
\PY{err}{│}   \PY{err}{├}\PY{err}{─}\PY{err}{─} \PY{n}{admin\PYZus{}users}\PY{o}{.}\PY{n}{html}    \PY{c+c1}{\PYZsh{} User list}
\PY{err}{│}   \PY{err}{└}\PY{err}{─}\PY{err}{─} \PY{n}{admin\PYZus{}user\PYZus{}edit}\PY{o}{.}\PY{n}{html} \PY{c+c1}{\PYZsh{} Role edit form}
\PY{err}{├}\PY{err}{─}\PY{err}{─} \PY{n}{sample\PYZus{}csv}\PY{o}{/}             \PY{c+c1}{\PYZsh{} Classroom datasets}
\PY{err}{│}   \PY{err}{├}\PY{err}{─}\PY{err}{─} \PY{n}{events}\PY{o}{.}\PY{n}{csv}
\PY{err}{│}   \PY{err}{├}\PY{err}{─}\PY{err}{─} \PY{n}{staff}\PY{o}{.}\PY{n}{csv}
\PY{err}{│}   \PY{err}{├}\PY{err}{─}\PY{err}{─} \PY{n}{customers}\PY{o}{.}\PY{n}{csv}
\PY{err}{│}   \PY{err}{└}\PY{err}{─}\PY{err}{─} \PY{n}{bookings}\PY{o}{.}\PY{n}{csv}
\PY{err}{├}\PY{err}{─}\PY{err}{─} \PY{n}{instance}\PY{o}{/}\PY{n}{config}\PY{o}{.}\PY{n}{py}      \PY{c+c1}{\PYZsh{} Config: DB URI, secrets, OAuth, JWT}
\PY{err}{├}\PY{err}{─}\PY{err}{─} \PY{n}{logs}\PY{o}{/}\PY{n}{app}\PY{o}{.}\PY{n}{log}            \PY{c+c1}{\PYZsh{} Rotating log file}
\PY{err}{├}\PY{err}{─}\PY{err}{─} \PY{n}{migrations}\PY{o}{/}             \PY{c+c1}{\PYZsh{} Flask\PYZhy{}Migrate scripts}
\PY{err}{└}\PY{err}{─}\PY{err}{─} \PY{n}{tests}\PY{o}{/}                  \PY{c+c1}{\PYZsh{} Unit + integration tests}
\end{Verbatim}
\end{tcolorbox}

    \subsubsection{🎯 Activity 2: Project
Structure}\label{activity-2-project-structure}

\begin{itemize}
\tightlist
\item
  Pick three files from the directory tree and describe their role. ✅
  Success criteria: correct file purpose, explains interactions, uses
  correct terminology.
\end{itemize}

    \subsection{3. Environment Setup - Steps to Create your Flask
PWA}\label{environment-setup---steps-to-create-your-flask-pwa}

\subsubsection{Virtual Environments}\label{virtual-environments}

\subsubsection{Requirements files}\label{requirements-files}

    In the cell below, \textless!pip\textgreater{} means ``run pip as a
shell command inside Jupyter Notebook''.

    \begin{tcolorbox}[breakable, size=fbox, boxrule=1pt, pad at break*=1mm,colback=cellbackground, colframe=cellborder]
\prompt{In}{incolor}{ }{\boxspacing}
\begin{Verbatim}[commandchars=\\\{\}]
\PY{c+c1}{\PYZsh{} Install all dev dependencies directly from the notebook}
\PY{o}{!}pip\PY{+w}{ }install\PY{+w}{ }\PYZhy{}r\PY{+w}{ }requirements\PYZhy{}dev.txt
\end{Verbatim}
\end{tcolorbox}

    \subsubsection{🎯 Activity 3: Environment
Setup}\label{activity-3-environment-setup}

\begin{itemize}
\tightlist
\item
  Run the install command.
\item
  Which tools in \texttt{requirements-dev.txt} are for security? ✅
  Success criteria: identifies Bandit, Semgrep, Talisman, explains dev
  vs prod differences.
\end{itemize}

    \subsection{4. Flask App Basics}\label{flask-app-basics}

    \begin{tcolorbox}[breakable, size=fbox, boxrule=1pt, pad at break*=1mm,colback=cellbackground, colframe=cellborder]
\prompt{In}{incolor}{ }{\boxspacing}
\begin{Verbatim}[commandchars=\\\{\}]
\PY{n}{app} \PY{o}{=} \PY{n}{Flask}\PY{p}{(}\PY{n+nv+vm}{\PYZus{}\PYZus{}name\PYZus{}\PYZus{}}\PY{p}{)}
\end{Verbatim}
\end{tcolorbox}

    \subsubsection{🎯 Activity 4: Flask
Basics}\label{activity-4-flask-basics}

\begin{itemize}
\tightlist
\item
  What does \texttt{app\ =\ Flask(\_\_name\_\_)} do? ✅ Success
  criteria: mentions app creation, entry point for routes, WSGI context.
\end{itemize}

    \subsection{5. Database Integration (Markdown →
Code)}\label{database-integration-markdown-code}

    \subsubsection{SQLAlchemy and why ORMs
matter.}\label{sqlalchemy-and-why-orms-matter.}

    \begin{tcolorbox}[breakable, size=fbox, boxrule=1pt, pad at break*=1mm,colback=cellbackground, colframe=cellborder]
\prompt{In}{incolor}{ }{\boxspacing}
\begin{Verbatim}[commandchars=\\\{\}]
\PY{c+c1}{\PYZsh{} To Do \PYZhy{}\PYZhy{}\PYZgt{} SQLAlchemy setup.}
\end{Verbatim}
\end{tcolorbox}

    \subsubsection{🎯 Activity 5: Database
Integration}\label{activity-5-database-integration}

\begin{itemize}
\tightlist
\item
  Why use SQLAlchemy instead of raw SQL? ✅ Success criteria: ORM
  abstraction, security via parameterization, reproducibility.
\end{itemize}

    \subsection{6. Models \& Constraints (Markdown →
Code)}\label{models-constraints-markdown-code}

    \begin{tcolorbox}[breakable, size=fbox, boxrule=1pt, pad at break*=1mm,colback=cellbackground, colframe=cellborder]
\prompt{In}{incolor}{ }{\boxspacing}
\begin{Verbatim}[commandchars=\\\{\}]
\PY{c+c1}{\PYZsh{} To Do \PYZhy{}\PYZhy{}\PYZgt{} Booking model with constraints.}
\end{Verbatim}
\end{tcolorbox}

    \subsubsection{🎯 Activity 6: Models \&
Constraints}\label{activity-6-models-constraints}

\begin{itemize}
\tightlist
\item
  Add a constraint so bookings never exceed 10 tickets.
\item
  Why is this important? ✅ Success criteria: correct CheckConstraint,
  explains business logic enforcement, connects to preventing abuse.
\end{itemize}

    \subsubsection{Explain entities (Users, Events,
Bookings).}\label{explain-entities-users-events-bookings.}

    \begin{tcolorbox}[breakable, size=fbox, boxrule=1pt, pad at break*=1mm,colback=cellbackground, colframe=cellborder]
\prompt{In}{incolor}{ }{\boxspacing}
\begin{Verbatim}[commandchars=\\\{\}]
\PY{c+c1}{\PYZsh{} show User and Booking models with constraints (security posture included).}
\end{Verbatim}
\end{tcolorbox}

    \subsubsection{To Do --\textgreater{} Show snippets of base.html,
manifest.json.}\label{to-do-show-snippets-of-base.html-manifest.json.}

    \subsection{7. Templates \& Static Assets
(Markdown)}\label{templates-static-assets-markdown}

    \subsubsection{To Do --\textgreater{} Explain manifest, service
worker.}\label{to-do-explain-manifest-service-worker.}

    \subsubsection{🎯 Activity: Templates \&
Assets}\label{activity-templates-assets}

\begin{itemize}
\tightlist
\item
  How do \texttt{manifest.json} and \texttt{service-worker.js} make this
  app a PWA? ✅ Success criteria: manifest metadata, offline caching,
  installability.
\end{itemize}

    \subsubsection{ToDo --\textgreater{} Jinja2 templates, service worker,
manifest.}\label{todo-jinja2-templates-service-worker-manifest.}

    \subsection{8. Authentication \& Authorization (Markdown →
Code)}\label{authentication-authorization-markdown-code}

    \subsubsection{To Do --\textgreater{} JWT config, role
guard.}\label{to-do-jwt-config-role-guard.}

    \subsubsection{🎯 Activity 8:
Authentication}\label{activity-8-authentication}

\begin{itemize}
\tightlist
\item
  What's the difference between OAuth login and JWT API tokens? ✅
  Success criteria: OAuth = identity provider login, JWT = stateless API
  access, expiry/roles.
\end{itemize}

    \subsubsection{Explain Google OAuth, JWT, role‑based
access.}\label{explain-google-oauth-jwt-rolebased-access.}

    \begin{tcolorbox}[breakable, size=fbox, boxrule=1pt, pad at break*=1mm,colback=cellbackground, colframe=cellborder]
\prompt{In}{incolor}{ }{\boxspacing}
\begin{Verbatim}[commandchars=\\\{\}]
\PY{c+c1}{\PYZsh{} To Do \PYZhy{}\PYZhy{}\PYZgt{} sample login route, JWT config, role guard decorator.}
\end{Verbatim}
\end{tcolorbox}

    \subsection{9. Admin Dashboard (Markdown →
Code)}\label{admin-dashboard-markdown-code}

    \subsubsection{explain purpose of admin
UI.}\label{explain-purpose-of-admin-ui.}

    \begin{tcolorbox}[breakable, size=fbox, boxrule=1pt, pad at break*=1mm,colback=cellbackground, colframe=cellborder]
\prompt{In}{incolor}{ }{\boxspacing}
\begin{Verbatim}[commandchars=\\\{\}]
\PY{c+c1}{\PYZsh{} To Do \PYZhy{}\PYZhy{}\PYZgt{} sample /admin route with role guard.}
\end{Verbatim}
\end{tcolorbox}

    \subsubsection{🎯 Activity 9: Admin
Dashboard}\label{activity-9-admin-dashboard}

\begin{itemize}
\tightlist
\item
  Why must admin routes be role-protected? Give an example. ✅ Success
  criteria: mentions least privilege, example of editing user roles,
  connects to audit safety.
\end{itemize}

    \subsection{10. Testing \& Security (Markdown →
Code)}\label{testing-security-markdown-code}

    \subsubsection{To Do --\textgreater{} explain importance of unit tests,
integration tests, and
SAST/DAST.}\label{to-do-explain-importance-of-unit-tests-integration-tests-and-sastdast.}

    \begin{tcolorbox}[breakable, size=fbox, boxrule=1pt, pad at break*=1mm,colback=cellbackground, colframe=cellborder]
\prompt{In}{incolor}{ }{\boxspacing}
\begin{Verbatim}[commandchars=\\\{\}]
\PY{c+c1}{\PYZsh{} To Do \PYZhy{}\PYZhy{}\PYZgt{} sample pytest test case.}
\end{Verbatim}
\end{tcolorbox}

    \begin{tcolorbox}[breakable, size=fbox, boxrule=1pt, pad at break*=1mm,colback=cellbackground, colframe=cellborder]
\prompt{In}{incolor}{ }{\boxspacing}
\begin{Verbatim}[commandchars=\\\{\}]
\PY{c+c1}{\PYZsh{} To Do \PYZhy{}\PYZhy{}\PYZgt{} run Bandit/Semgrep inside notebook (!bandit \PYZhy{}r flask\PYZus{}pwa\PYZus{}app).}
\end{Verbatim}
\end{tcolorbox}

    \begin{tcolorbox}[breakable, size=fbox, boxrule=1pt, pad at break*=1mm,colback=cellbackground, colframe=cellborder]
\prompt{In}{incolor}{ }{\boxspacing}
\begin{Verbatim}[commandchars=\\\{\}]
\PY{o}{!}bandit\PY{+w}{ }\PYZhy{}r\PY{+w}{ }flask\PYZus{}pwa\PYZus{}app
\end{Verbatim}
\end{tcolorbox}

    \subsubsection{🎯 Activity 10: Security
Testing}\label{activity-10-security-testing}

\begin{itemize}
\tightlist
\item
  Run Bandit. What type of issues does it detect? ✅ Success criteria:
  runs tool, identifies issue type (hardcoded secrets, unsafe
  functions), explains importance.
\end{itemize}

    \subsection{11. Deployment Preparation (Markdown →
Code)}\label{deployment-preparation-markdown-code}

    \begin{tcolorbox}[breakable, size=fbox, boxrule=1pt, pad at break*=1mm,colback=cellbackground, colframe=cellborder]
\prompt{In}{incolor}{ }{\boxspacing}
\begin{Verbatim}[commandchars=\\\{\}]
\PY{c+c1}{\PYZsh{} To Do \PYZhy{}\PYZhy{}\PYZgt{} Dockerfile snippet.}
\end{Verbatim}
\end{tcolorbox}

    \subsubsection{🎯 Activity 11: Deployment
Preparation}\label{activity-11-deployment-preparation}

\begin{itemize}
\tightlist
\item
  Why use Docker instead of running Flask directly? ✅ Success criteria:
  reproducibility, environment consistency, CI/CD pipelines.
\end{itemize}

    \subsubsection{To Do --\textgreater{} explain
containerization.}\label{to-do-explain-containerization.}

    \begin{tcolorbox}[breakable, size=fbox, boxrule=1pt, pad at break*=1mm,colback=cellbackground, colframe=cellborder]
\prompt{In}{incolor}{ }{\boxspacing}
\begin{Verbatim}[commandchars=\\\{\}]
\PY{c+c1}{\PYZsh{} To Do \PYZhy{}\PYZhy{}\PYZgt{} Dockerfile snippet.}
\end{Verbatim}
\end{tcolorbox}

    \subsubsection{🎯 Activity 11: Deployment
Prep}\label{activity-11-deployment-prep}

\begin{itemize}
\tightlist
\item
  Why use Docker instead of running Flask directly? ✅ Success criteria:
  reproducibility, environment consistency, CI/CD pipelines.
\end{itemize}

    \subsubsection{To Do --\textgreater{} explain Fly.io
config.}\label{to-do-explain-fly.io-config.}

    \begin{tcolorbox}[breakable, size=fbox, boxrule=1pt, pad at break*=1mm,colback=cellbackground, colframe=cellborder]
\prompt{In}{incolor}{ }{\boxspacing}
\begin{Verbatim}[commandchars=\\\{\}]
\PY{c+c1}{\PYZsh{} To Do \PYZhy{}\PYZhy{}\PYZgt{} fly.toml snippet.}
\end{Verbatim}
\end{tcolorbox}

    \subsection{12. CI/CD Automation (Markdown →
Code)}\label{cicd-automation-markdown-code}

    \subsubsection{To Do --\textgreater{} Explain
workflows.}\label{to-do-explain-workflows.}

    \subsubsection{🎯 Activity 12: CI/CD}\label{activity-12-cicd}

\begin{itemize}
\tightlist
\item
  What happens if a test fails in GitHub Actions? ✅ Success criteria:
  pipeline stops, feedback loop, quality assurance.
\end{itemize}

    \subsubsection{To Do --\textgreater{} explain GitHub Actions
workflow.}\label{to-do-explain-github-actions-workflow.}

    \begin{tcolorbox}[breakable, size=fbox, boxrule=1pt, pad at break*=1mm,colback=cellbackground, colframe=cellborder]
\prompt{In}{incolor}{ }{\boxspacing}
\begin{Verbatim}[commandchars=\\\{\}]
\PY{c+c1}{\PYZsh{} To Do \PYZhy{}\PYZhy{}\PYZgt{} show ci.yml and deploy.yml.}
\end{Verbatim}
\end{tcolorbox}

    \subsection{13. Going Live (Markdown)}\label{going-live-markdown}

    \subsubsection{To Do --\textgreater{} Explain environment variables,
secrets, TLS, production
DB.}\label{to-do-explain-environment-variables-secrets-tls-production-db.}

    \subsubsection{🎯 Activity 13: Going Live}\label{activity-13-going-live}

\begin{itemize}
\tightlist
\item
  Why must secrets be stored in environment variables? ✅ Success
  criteria: explains risk of hardcoding, mentions rotation/audit,
  compliance.
\end{itemize}

    \begin{tcolorbox}[breakable, size=fbox, boxrule=1pt, pad at break*=1mm,colback=cellbackground, colframe=cellborder]
\prompt{In}{incolor}{ }{\boxspacing}
\begin{Verbatim}[commandchars=\\\{\}]
\PY{c+c1}{\PYZsh{} To Do \PYZhy{}\PYZhy{}\PYZgt{} Show how to run docker\PYZhy{}compose up locally.}
\end{Verbatim}
\end{tcolorbox}

    \subsection{14. Reflection \& Wrap‑Up
(Markdown)}\label{reflection-wrapup-markdown}

    \subsection{✅ Reflection}\label{reflection}

\begin{itemize}
\tightlist
\item
  List three security measures we added and explain their purpose. ✅
  Success criteria: identifies measures (CSP, rate limiting, DB
  constraints), explains clearly, connects to overall posture.
\end{itemize}

    \subsubsection{To Do --\textgreater{} Prompt students: ``What security
risks did we mitigate? How does CI/CD enforce
quality?''}\label{to-do-prompt-students-what-security-risks-did-we-mitigate-how-does-cicd-enforce-quality}

    \subsubsection{To Do --\textgreater{} Encourage them to connect the
Theme Park scenario to real‑world
apps.}\label{to-do-encourage-them-to-connect-the-theme-park-scenario-to-realworld-apps.}

    \subsubsection{Step 1: Initialize Flask
App}\label{step-1-initialize-flask-app}

    \begin{tcolorbox}[breakable, size=fbox, boxrule=1pt, pad at break*=1mm,colback=cellbackground, colframe=cellborder]
\prompt{In}{incolor}{ }{\boxspacing}
\begin{Verbatim}[commandchars=\\\{\}]
\PY{k+kn}{from}\PY{+w}{ }\PY{n+nn}{flask}\PY{+w}{ }\PY{k+kn}{import} \PY{n}{Flask}
\PY{n}{app} \PY{o}{=} \PY{n}{Flask}\PY{p}{(}\PY{n+nv+vm}{\PYZus{}\PYZus{}name\PYZus{}\PYZus{}}\PY{p}{)}
\end{Verbatim}
\end{tcolorbox}

    \subsubsection{Step 2: Configure your
Database}\label{step-2-configure-your-database}

    \begin{tcolorbox}[breakable, size=fbox, boxrule=1pt, pad at break*=1mm,colback=cellbackground, colframe=cellborder]
\prompt{In}{incolor}{ }{\boxspacing}
\begin{Verbatim}[commandchars=\\\{\}]
\PY{k+kn}{from}\PY{+w}{ }\PY{n+nn}{flask\PYZus{}sqlalchemy}\PY{+w}{ }\PY{k+kn}{import} \PY{n}{SQLAlchemy}
\PY{n}{app}\PY{o}{.}\PY{n}{config}\PY{p}{[}\PY{l+s+s1}{\PYZsq{}}\PY{l+s+s1}{SQLALCHEMY\PYZus{}DATABASE\PYZus{}URI}\PY{l+s+s1}{\PYZsq{}}\PY{p}{]} \PY{o}{=} \PY{l+s+s1}{\PYZsq{}}\PY{l+s+s1}{sqlite:///themepark.db}\PY{l+s+s1}{\PYZsq{}}
\PY{n}{db} \PY{o}{=} \PY{n}{SQLAlchemy}\PY{p}{(}\PY{n}{app}\PY{p}{)}
\end{Verbatim}
\end{tcolorbox}

    \subsubsection{Step 3: Define Models}\label{step-3-define-models}

    This is where we set up:the structure of the database, the tables (also
called entities), how they are related to each other, the data
dictionary, the field names (also called attributes), field length,
field data type and any other useful information. The database will
contain a table of customers, staff, events and bookings.

    \begin{tcolorbox}[breakable, size=fbox, boxrule=1pt, pad at break*=1mm,colback=cellbackground, colframe=cellborder]
\prompt{In}{incolor}{ }{\boxspacing}
\begin{Verbatim}[commandchars=\\\{\}]
\PY{k}{class}\PY{+w}{ }\PY{n+nc}{User}\PY{p}{(}\PY{n}{db}\PY{o}{.}\PY{n}{Model}\PY{p}{)}\PY{p}{:}
    \PY{n+nb}{id} \PY{o}{=} \PY{n}{db}\PY{o}{.}\PY{n}{Column}\PY{p}{(}\PY{n}{db}\PY{o}{.}\PY{n}{Integer}\PY{p}{,} \PY{n}{primary\PYZus{}key}\PY{o}{=}\PY{k+kc}{True}\PY{p}{)}
    \PY{n}{email} \PY{o}{=} \PY{n}{db}\PY{o}{.}\PY{n}{Column}\PY{p}{(}\PY{n}{db}\PY{o}{.}\PY{n}{String}\PY{p}{(}\PY{l+m+mi}{255}\PY{p}{)}\PY{p}{,} \PY{n}{unique}\PY{o}{=}\PY{k+kc}{True}\PY{p}{)}
    \PY{n}{role} \PY{o}{=} \PY{n}{db}\PY{o}{.}\PY{n}{Column}\PY{p}{(}\PY{n}{db}\PY{o}{.}\PY{n}{String}\PY{p}{(}\PY{l+m+mi}{50}\PY{p}{)}\PY{p}{,} \PY{n}{default}\PY{o}{=}\PY{l+s+s2}{\PYZdq{}}\PY{l+s+s2}{student}\PY{l+s+s2}{\PYZdq{}}\PY{p}{)}
\end{Verbatim}
\end{tcolorbox}

    \subsubsection{Step 4: Add Templates \& Static
Assets}\label{step-4-add-templates-static-assets}

\begin{itemize}
\tightlist
\item
  \texttt{base.html} → layout with navbar, footer, version injection.
\item
  \texttt{service-worker.js} → caches assets for offline use.
\item
  \texttt{manifest.json} → defines PWA metadata.
\end{itemize}

\subsubsection{Step 5: Authentication}\label{step-5-authentication}

\begin{itemize}
\tightlist
\item
  Google OAuth via Authlib.
\item
  Flask-Login for sessions.
\item
  JWT for API endpoints.
\end{itemize}

\subsubsection{Step 6: Admin UI}\label{step-6-admin-ui}

\begin{itemize}
\tightlist
\item
  \texttt{/admin} → dashboard.
\item
  \texttt{/admin/users} → list users.
\item
  \texttt{/admin/users/\textless{}id\textgreater{}/edit} → edit roles.
\end{itemize}

\subsubsection{Step 7: Testing}\label{step-7-testing}

\begin{itemize}
\tightlist
\item
  Pytest fixtures (\texttt{conftest.py}).
\item
  Unit tests for models, routes, JWT.
\item
  Integration tests for login/logout cycle.
\end{itemize}

    \subsection{4. Going Live: Deployment
Steps}\label{going-live-deployment-steps}

\subsubsection{Step 1: Containerization}\label{step-1-containerization}

    \begin{tcolorbox}[breakable, size=fbox, boxrule=1pt, pad at break*=1mm,colback=cellbackground, colframe=cellborder]
\prompt{In}{incolor}{ }{\boxspacing}
\begin{Verbatim}[commandchars=\\\{\}]
\PY{o}{\PYZpc{}\PYZpc{}}\PY{k}{bash}
cat \PYZgt{} Dockerfile \PYZlt{}\PYZlt{}\PYZsq{}EOF\PYZsq{}
FROM python:3.11\PYZhy{}slim
WORKDIR /app
COPY requirements.txt .
RUN pip install \PYZhy{}r requirements.txt
COPY . .
CMD [\PYZdq{}gunicorn\PYZdq{}, \PYZdq{}\PYZhy{}b\PYZdq{}, \PYZdq{}0.0.0.0:8080\PYZdq{}, \PYZdq{}app:app\PYZdq{}]
EOF
\end{Verbatim}
\end{tcolorbox}

    \subsubsection{Step 2: Fly.io Config}\label{step-2-fly.io-config}

    \begin{tcolorbox}[breakable, size=fbox, boxrule=1pt, pad at break*=1mm,colback=cellbackground, colframe=cellborder]
\prompt{In}{incolor}{ }{\boxspacing}
\begin{Verbatim}[commandchars=\\\{\}]
\PY{o}{\PYZpc{}\PYZpc{}}\PY{k}{bash}
cat \PYZgt{} fly.toml \PYZlt{}\PYZlt{}\PYZsq{}EOF\PYZsq{}
app = \PYZdq{}themepark\PYZhy{}pwa\PYZdq{}
primary\PYZus{}region = \PYZdq{}syd\PYZdq{}
[build]
  dockerfile = \PYZdq{}Dockerfile\PYZdq{}
EOF
\end{Verbatim}
\end{tcolorbox}

    \subsubsection{Step 3: CI/CD}\label{step-3-cicd}

\begin{itemize}
\tightlist
\item
  \texttt{.github/workflows/ci.yml} → linting, tests, coverage.
\item
  \texttt{.github/workflows/deploy.yml} → auto-deploy to Heroku or
  Fly.io.
\end{itemize}

\subsubsection{Step 4: Production
Modifications}\label{step-4-production-modifications}

\begin{itemize}
\tightlist
\item
  Switch DB URI to Postgres (\texttt{SQLALCHEMY\_DATABASE\_URI}).
\item
  Set \texttt{SESSION\_COOKIE\_SECURE=True}.
\item
  Use environment variables for secrets (\texttt{SECRET\_KEY}, OAuth
  keys).
\item
  Configure HTTPS (TLS via Fly.io or Certbot).
\item
  Add logging to external service if scaling.
\end{itemize}

    \subsection{✅ Summary}\label{summary}

This notebook demonstrates: - \textbf{Purpose}: secure, classroom-ready
PWA. - \textbf{Design logic}: modular, auditable, reproducible. -
\textbf{Steps to build}: Flask app, DB, auth, templates, tests. -
\textbf{Deployment}: Docker, Fly.io, CI/CD, production hardening.

Students can run cells, inspect files, and follow deployment steps to
see how a Flask PWA goes from \textbf{local dev → production live app}.

    \begin{tcolorbox}[breakable, size=fbox, boxrule=1pt, pad at break*=1mm,colback=cellbackground, colframe=cellborder]
\prompt{In}{incolor}{ }{\boxspacing}
\begin{Verbatim}[commandchars=\\\{\}]

\end{Verbatim}
\end{tcolorbox}


    % Add a bibliography block to the postdoc
    
    
    
\end{document}
